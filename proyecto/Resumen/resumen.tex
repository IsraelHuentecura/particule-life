\rhead{}
\lhead{}
\renewcommand{\headrulewidth}{0pt}
\addcontentsline{toc}{chapter}{\textbf{Abstract}} 
\begin{center}
    \Large
    \textbf{\newtitle}
    
    \large
    \vspace{0.4cm}
    \newauthor
    \vspace{0.9cm}
    
    \textbf{Abstract}\\
\end{center}


Quantum control (QC) allows directing quantum systems to exploit all their advantages, such as state-to-state transfer, quantum sensing, preparation of entangled states, and quantum register. QC is of great interest to physicists due to its atomic, optical, mechanical, and solid-state applications. Optimal control can be achieved by different techniques. In this thesis, we manage to control a solid-state system formed by a nitrogen-vacancy center NV$^{-}$ and two $^{13}$C nuclear spins in the microwave (MW) regime. NV$^{-}$ is used as an ancilla for the control and initialization of $^{13}$C nuclear spins. Using this model, we make a decoherence-protected quantum register, where the quantum information is processed in two logical states formed by the two $^{13}$C nuclear spins. The quantum register was manipulated by tuning an off-axis magnetic field that enables non-nuclear-spin-preserving transitions that we use for preparing and manipulating the register through stimulated Raman adiabatic passage (STIRAP). Furthermore, we consider more elaborated sequences to improve simultaneous control over the system yielding decreased gate time. On the other hand, we propose this system as an experimental platform to study dynamical quantum phase transitions (DQPT). DQPT has attracted a great deal of attention in recent years because it is a stage for studying the dynamics of many-body quantum systems. We show that $^{13}$C nuclear spins undergo DQPT by appropriately choosing the relationship between an external magnetic field's transverse and longitudinal components. In addition, we propose a quenched dynamics that originates from the rotation of the central electron spin (NV$^{-}$), which controls the DQPT relying on the anisotropy of the hyperfine coupling. In order to optimize the quantum control for multi-level spin systems, we introduce a computational method based on physics-informed neural networks (PINNs). We applied this PINNs scheme to three- and four-level systems to discover the optimal control fields for the population transfer problem. By introducing a loss function into the neural network's architecture, we can achieve high probabilities while minimizing the transfer time compared to other adiabatic and non-adiabatic control methods.

\newpage
\begin{center}
    \Large
    \textbf{Aplicaciones de control cuántico a spins de estado sólido}
    
    \large
    \vspace{0.4cm}
    \newauthor
    \vspace{0.9cm}
    
    \textbf{Resumen}\\
\end{center}

El control cuántico (CC) permite dirigir los sistemas cuánticos para explotar todas sus ventajas, como la transferencia de estado a estado, la detección cuántica, la preparación de estados entrelazados y el registro cuántico. CC es de gran interés para los físicos debido a sus aplicaciones atómicas, ópticas, mecánicas y de estado sólido. El control óptimo se puede lograr mediante diferentes técnicas. En esta tesis logramos controlar un sistema de estado sólido formado por un centro vacante de nitrógeno NV$^{-}$ y dos spins nucleares $^{13}$C en el régimen de las microondas (MW). El NV$^{-}$ se utiliza como auxiliar para el control y la inicialización de los spins nucleares de $^{13}$C. Usando este modelo, creamos un registro cuántico protegido por decoherencia, donde la información cuántica se procesa en dos estados lógicos formados por los dos spins nucleares $^{13}$C. El registro cuántico se manipuló sintonizando un campo magnético fuera del eje que permite transiciones que conservan el spín no nuclear que usamos para preparar y manipular el registro a través del paso adiabático Raman estimulado (STIRAP). Además, consideramos secuencias más elaboradas para mejorar el control simultáneo sobre el sistema, lo que reduce el tiempo de las compuertas. Por otro lado, proponemos este sistema como una plataforma experimental para estudiar transiciones de fase cuánticas dinámicas (TFCD). TFCD ha llamado mucho la atención en los últimos años porque es un escenario para estudiar la dinámica de los sistemas cuánticos de muchos cuerpos. Mostramos que los spins nucleares de $^{13}$C experimentan TFCD eligiendo apropiadamente la relación entre los componentes transversales y longitudinales de un campo magnético externo. Además, proponemos una dinámica de extinción que se origina a partir de la rotación del spín central del electrón (NV$^{-}$), que controla el TFCD basándose en la anisotropía del acoplamiento hiperfino. Con el fin de optimizar el control cuántico para sistemas de   multi-niveles de spín, presentamos un método computacional basado en redes neuronales informadas por la física (RNIF). Aplicamos este esquema de RNIFs a sistemas de tres y cuatro niveles para obtener los campos de control óptimos para el problema de transferencia de población. Al introducir una función de costo en la arquitectura de la red neuronal, podemos lograr altas probabilidades y minimizar el tiempo de transferencia en comparación con otros métodos de control adiabáticos y no adiabáticos.